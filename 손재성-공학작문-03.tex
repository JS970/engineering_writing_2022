\documentclass{article}
\usepackage{kotex}
\usepackage{csquotes}
\usepackage{indentfirst}

\title{Android를 통해 알아보는 Rust의 왕위계승}

\author{전기컴퓨터공학부 정보컴퓨터공학 전공 201824501 손재성\footnote{son73097@pusan.ac.kr}}

\date{2022.10.3}

\begin{document}
\maketitle
\section{C/C++의 문제점}
안드로이드 운영체제는 리눅스 기반의 모바일 운영체제로, 터치스크린 인터페이스를 가지는 모바일 기기들에 중점을 두어 설계되었다.
현재 수많은 모바일 기기에서 광범위하게 사용되는 안드로이드 운영체제상의 응용프로그램들은 주로 Java/Kotlin으로 개발된다.
반면에, 안드로이드 내부에서 동작하는 커널이나 디바이스 드라이버와 같은 low-level코드는 운영체제 레벨에서 시스템의 리소스와 하드웨어를 직접 제어해야 하며 예측 가능한
퍼포먼스가 요구되기 때문에 Java/Kotlin이 아닌 C/C++로 작성된다. 안드로이드의 어플리케이션 레벨에서 사용되는 Java/Kotlin은 ART에서 메모리를 관리해 주기 때문에 메모리 관련 이슈가 적은 편이지만,
C/C++은 메모리 안정성을 보장하기가 힘든 특성으로 인해 관련 문제가 빈번하게 발생한다.
\section{Rust}
Rust는 이러한 기존의 C/C++등의 언어와 달리, 컴파일러 언어이면서도 높은 메모리 안정성을 제공한다. Rust는 Null pointer, dangling pointer, data races를 허용하지 않는다.
데이터 값들은 오직 정해진 형식을 통해서만 초기화되며 모든 값들이 이미 초기화된 상태이어야 한다. 특정 자료 구조에서 사용하는 NULL포인터와 같은 역할을 수행하기 위해 Rust코어 라이브러리는 포인터에 어떤 값 또는 'None'이 있는지 테스트하는 데 사용하는 옵션을 제공한다. 이러한 특징을 바탕으로 메모리 안정성을 제공하는 Rust는 시스템 레벨에서 C/C++을 대체할 수 있는 언어로 지목된다.\footnote{https://en.wikipedia.org/wiki/Rust\_(programming\_language)

(assessed Oct. 4. 2022)}
\section{Google의 동향}
최근 구글은 안드로이드에 리포트된 버그들 중 최근에 리포트된 것들부터 C/C++에서 Rust로 변경하는 작업을 시작했다. 최근에 리포트된 것들부터 작업을 시작한 이유는 오래된 버그는 굳이 
고칠 필요가 없는 중대한 버그가 아닐 가능성이 높기 때문이다. 하지만 이런 식의 대체가 긍정적인 측면만 있는 것은 아니다. Rust가 메모리 안정성을 제공하면서 성능도 보장한다는 큰 장점이 있지만,
수십 년간 그 위치를 지켜온 C/C++에 비해서 생태계가 부실하다. 또한 메모리 안정성을 제공한다는 특징 때문에, 실제 개발자 시점에서는 제약이 많아 불편하게 느껴진다.
그럼에도 불구하고 구글과 같은 거대 기업에서 C/C++에서 Rust로의 교체를 시작한 것에 대해, 
본 글을 쓰기 위해 참고한 블로그 글의 저자는 "언젠가는 low-level에서 Rust가 C/C++의 왕좌를 위협할 날이 오지 않을까 싶다"라는 소감을 밝혔다.\footnote{https://brunch.co.kr/@advisor/31(assessed Oct. 4. 2022)}
\section{내 생각}
나온 지 얼마 되지 않았으며 시스템 레벨에서 사용되고 C/C++을 대체할 수 있는 언어라는 명성을 접하게 되어 처음 Rust에 대해 관심을 갖기 시작했고, 안드로이드 시스템 프로그래밍은 아니지만, 간단한 CLI 프로그램을 작성하는 작업을 해 보았는데, 다른 프로그래밍 언어에 비해 변수 선언, 함수 생성 등 많은 부분에서 생소함을 경험했던 기억이 있다. 본인은 처음 프로그래밍을 C/C++로 접했으며, 아직 Rust뿐만 아니라 Java/Kotlin등의 실무에서 널리 사용되는 언어를 활용한 프로그래밍 경험이 많지 않아서 아직은 여타 언어에 비해 C/C++이 더 익숙하다. 하지만 C/C++을 세상에 나온 지 이미 수십 년이 지난 언어이며, 나온 지 수십 년이 지난 현재까지도 고질적인 메모리 관리 문제를 겪고 있는 것은 익히 들어왔다. 이런 동향 속에서 수십 년간 시스템 레벨에서의 위치를 지켜온 C/C++을 대체할 언어로 지목되는 Rust가 안드로이드에서부터 서서히 정말로 C/C++을 대체하고 있는 것이 굉장히 흥미롭다.
\end{document}
